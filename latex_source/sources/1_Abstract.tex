\section*{Abstract}
The interaction between T-cell receptors (TCRs) and peptide major histocompatibility complexes (pMHC) is essential for the function of the adaptive immune system. Predictions of TCR-pMHC interactions have proven to be a challenge, but accurate predictions have the potential to further vaccine development and cancer treatments. 

Here, we will model TCR-pMHC interactions using deep neural networks using a previously created benchmark dataset containing both sequences and predicted energy features. The best performance was obtained using only the peptide and paired complementarity-determining regions (CDR)3 sequences. The remaining sequences and all energy features were unable to provide additional information for models trained on the full dataset.

Furthermore, we will show that attention mechanisms can capture sequence variations between binding and non-binding TCR sequences. Long short-term memory (LSTM) coupled with attention mechanisms also improved performance over convolutional neural networks (CNNs) and LSTMs without attention.

Pan-specific and peptide-specific models showed comparable performances for predicting TCR-pMHC interactions, with pan-specific models performing slightly better when less data was used for training. Pan-specific models were also better at predicting peptide specificity and can be used to predict multiple peptide with the same model. TCR-pMHC binding predictions are feasible on peptides with a substantial number of observations. However, the models' ability to generalize to other peptides is most likely limited by a lack of data and additional TCRs for current and new peptides are needed for a truly pan-specific model.

